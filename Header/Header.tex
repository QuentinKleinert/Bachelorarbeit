%allgemeine Formatangaben
\documentclass[
 a4paper, 										% Papierformat
 12pt,												% Schriftgröße
 ngerman, 										% für Umlaute, Silbentrennung etc.
 titlepage,										% es wird eine Titelseite verwendet
 bibliography=totoc,					% Literaturverzeichnis im Inhaltsverzeichnis aufführen
 listof=totoc,								% Verzeichnisse im Inhaltsverzeichnis aufführen
 oneside, 										% einseitiges Dokument
 captions=nooneline,					% einzeilige Gleitobjekttitel ohne Sonderbehandlung wie mehrzeilige Gleitobjekttitel behandeln
 numbers=noenddot,						% Überschriften-??Nummerierung ohne Punkt am Ende
 parskip=half									% zwischen Absätzen wird eine halbe Zeile eingefügt
 ]{scrbook}


\newcommand{\titel}{Nutzung von Large Language Models zur semantischen Bearbeitung und Abfrage RDF-basierter Datenbanken – eine SPARQL-Schnittstelle mit natürlicher Sprache}
\newcommand{\abschlussart}{Bachelor of Science (B.Sc.)}
\newcommand{\arbeit}{Bachelorarbeit}
\newcommand{\hochschule}{Hochschule für Technik, Wirtschaft und Kultur Leipzig}
\newcommand{\fakultaet}{Fakultaet Informatik und Medien}
\newcommand{\autor}{Quentin Kleinert}
\newcommand{\studiengang}{Bachelorstudiengang Informatik}
\newcommand{\matrikelnr}{82665}
\newcommand{\erstgutachter}{Prof. Dr. rer. nat. Thomas Riechert}
\newcommand{\zweitgutachter}{Jonas Gwodz}
\newcommand{\ort}{Leipzig}
 			% einbinden von persönlichen Daten

% ============ Sprache & Kodierung ============
\usepackage[ngerman]{babel}
\usepackage[utf8]{inputenc}     % statt latin1
\usepackage[T1]{fontenc}
\usepackage{textcomp}           % Euro & Co. (bei neuen LaTeX-Installationen oft entbehrlich)

% \RequirePackage[ngerman=ngerman-x-latest]{hyphsubst} % meist nicht mehr nötig

% ============ Mathe ============
\usepackage{amsmath,amsfonts}

% ============ Layout ============
\usepackage{setspace}
\usepackage{geometry}

% ============ Grafiken & Float-Handling ============
\usepackage{graphicx}
\usepackage{svg}
\usepackage{wrapfig}            % statt veraltetem floatflt
% \usepackage{pstricks}         % -> vermeiden bei pdfLaTeX; alternativ TikZ:
% \usepackage{tikz}

% ============ Farben ============
\usepackage[dvipsnames]{xcolor} % statt color
% Beispiel-Farben, falls du sie in hyperref nutzt:
\definecolor{InterneLinkfarbe}{RGB}{0,0,120}
\definecolor{ExterneLinkfarbe}{RGB}{0,100,0}
\definecolor{GrayBlue}{RGB}{90,105,125}

% ============ Tabellen ============
\usepackage{array}
\usepackage{longtable}
\usepackage{booktabs}         
\usepackage{tabularx}

% ============ Quellcode ============
\usepackage{verbatim}
\usepackage{moreverb}
\usepackage{listingsutf8}
\usepackage{listings}
\lstloadlanguages{Java,HTML,bash}
\lstdefinelanguage{SPARQL}{
  morekeywords={SELECT,ASK,CONSTRUCT,DESCRIBE,WHERE,FROM,GRAPH,OPTIONAL,FILTER,
    LIMIT,OFFSET,ORDER,BY,GROUP,HAVING,VALUES,UNION,BIND,SERVICE,MINUS,INSERT,
    DELETE,DATA,WITH,USING,LOAD,CLEAR,CREATE,DROP,ADD,MOVE,COPY},
  sensitive=true,
  morecomment=[l]{\#},
  morestring=[b]"
}
\lstset{
  language=SPARQL,
  inputencoding=utf8,
  frame=tb,
  basicstyle=\footnotesize\ttfamily,
  showstringspaces=false,
  keywordstyle=\bfseries,
  commentstyle=\color{GrayBlue}\itshape,
  xleftmargin=\parindent,
  xrightmargin=0pt,
  framexleftmargin=\parindent,
  framexrightmargin=0pt,
  aboveskip=\bigskipamount,
  belowskip=\bigskipamount,
  breaklines=true,
  literate=
    {ä}{{\"a}}1 {ö}{{\"o}}1 {ü}{{\"u}}1 {Ä}{{\"A}}1 {Ö}{{\"O}}1 {Ü}{{\"U}}1
    {ß}{{\ss}}1 {–}{{-}}1 {—}{{-}}1 {“}{{\textquotedblleft}}1 {”}{{\textquotedblright}}1
    {„}{{\glqq}}1 {‚}{{\glq}}1 {…}{{\ldots}}1
}

\lstdefinelanguage{turtle}{
  morekeywords={@prefix,@base,a},
  sensitive=true,
  morecomment=[l]{\#},
  morestring=[b]""
}

% ============ Zitate ============
\usepackage[autostyle=true]{csquotes}
\usepackage[numbers,sort&compress]{natbib}

% ============ Sonstiges ============
\usepackage{xspace}
\usepackage{caption}
\usepackage{scrhack}
\usepackage{microtype}

% ============ Links (nahe ans Ende) ============
\usepackage[
  pdftex,
  colorlinks=true,
  linkcolor=InterneLinkfarbe,
  urlcolor=ExterneLinkfarbe
]{hyperref}
% \usepackage[all]{hypcap}  % i. d. R. nicht mehr nötig
% \usepackage{url}          % wird von hyperref abgedeckt

% ============ Glossar (nach hyperref laden) ============
\usepackage[
  nonumberlist,
  acronym,
  toc
]{glossaries}

\svgpath{{Abbildungen/}}					% einbinden der verwendeten Latex-Pakete


\onehalfspacing 							% 1,5facher Zeilenabstand

\definecolor{InterneLinkfarbe}{rgb}{0.1,0.1,0.3} 	% Farbliche Absetzung von externen Links
\definecolor{ExterneLinkfarbe}{rgb}{0.1,0.1,0.7}	% Farbliche Absetzung von internen Links

% Einstellungen für Fußnoten:
\captionsetup{font=footnotesize,labelfont=sc,singlelinecheck=true,margin={5mm,5mm}}

% Stil der Quellenangabe
\bibliographystyle{alphadin}

%Ausschluss von Schusterjungen
\clubpenalty = 10000
%Ausschluss von Hurenkindern
\widowpenalty = 10000

% Befehle, die Umlaute ausgeben, führen zu Fehlern, wenn sie hyperref als Optionen übergeben werden
\hypersetup{
%    pdftitle={\titel \untertitel},
%    pdfauthor={\autor},
%    pdfcreator={\autor},
%    pdfsubject={\titel \untertitel},
%    pdfkeywords={\titel \untertitel},
}

% Beispiel für eine Listings-Codeumbebungen
% Bei mehreren Definitionen empfielt sich das auslagern in eine externe Datei
\lstloadlanguages{Java,HTML}
\lstset{
	frame=tb,
	framesep=5pt,
	basicstyle=\footnotesize\ttfamily,
	showstringspaces=false,
	keywordstyle=\ttfamily\bfseries\color{CadetBlue},
	identifierstyle=\ttfamily,
	stringstyle=\ttfamily\color{OliveGreen},
	commentstyle=\color{GrayBlue},
	rulecolor=\color{Gray},
	xleftmargin=5pt,
	xrightmargin=5pt,
	aboveskip=\bigskipamount,
	belowskip=\bigskipamount
} 

%Den Punkt am Ende jeder Beschreibung deaktivieren
\renewcommand*{\glspostdescription}{}

% Empfehlung: Abkuerzungsverzeichnis und Glossar sind in Graduierunsarbeiten
% nicht zwingend notwendig

% %Glossar-Befehle anschalten
% \makeglossaries
% \glsenablehyper
% \newacronym{abac}{ABAC}{Attribute-Based Access Control}
\newacronym{acl}{ACL}{Access Control List}
\newacronym{api}{API}{Application Programming Interface}
\newacronym{authn}{AuthN}{Authentication (Authentifizierung von Nutzerinnen und Nutzern)}
\newacronym{authz}{AuthZ}{Authorization (Autorisierung von Zugriffsrechten)}
\newacronym{bgp}{BGP}{Basic Graph Pattern}
\newacronym{cbac}{CBAC}{Context-Based Access Control}
\newacronym{clm}{CLM}{Causal Language Modeling}
\newacronym{cot}{CoT}{Chain of Thought}
\newacronym{crud}{CRUD}{Create, Read, Update, Delete}
\newacronym{dac}{DAC}{Discretionary Access Control}
\newacronym{dcat}{DCAT}{Data Catalog Vocabulary}
\newacronym{dp}{DP}{Differential Privacy}
\newacronym{dpsgd}{DP-SGD}{Differentially Private Stochastic Gradient Descent}
\newacronym{dpv}{DPV}{Data Privacy Vocabulary}
\newacronym{dqv}{DQV}{Data Quality Vocabulary}
\newacronym{dsgvo}{DSGVO}{Datenschutz-Grundverordnung der Europäischen Union}
\newacronym{eugrch}{EU-GRCh}{Charta der Grundrechte der Europäischen Union}
\newacronym{etl}{ETL}{Extract, Transform, Load}
\newacronym{gpt}{GPT}{Generative Pre-trained Transformer}
\newacronym{hmac}{HMAC}{Hash-based Message Authentication Code}
\newacronym{http}{HTTP}{Hypertext Transfer Protocol}
\newacronym{https}{HTTPS}{Hypertext Transfer Protocol Secure}
\newacronym{iri}{IRI}{Internationalized Resource Identifier}
\newacronym{json}{JSON}{JavaScript Object Notation}
\newacronym{jsonld}{JSON-LD}{JSON for Linked Data}
\newacronym{kg}{KG}{Knowledge Graph}
\newacronym{kgqa}{KGQA}{Knowledge Graph Question Answering}
\newacronym{lcquad}{LC-QuAD}{Large-scale Complex Question Answering Dataset}
\newacronym{llm}{LLM}{Large Language Model}
\newacronym{mac}{MAC}{Mandatory Access Control}
\newacronym{nlzwo}{NL2SPARQL}{Plattform zur Übersetzung natürlicher Sprache in SPARQL-Updates}
\newacronym{owl}{OWL}{Web Ontology Language}
\newacronym{pii}{PII}{Personally Identifiable Information}
\newacronym{ppo}{PPO}{Proximal Policy Optimization}
\newacronym{provo}{PROV-O}{PROV Ontology}
\newacronym{qa}{QA}{Question Answering}
\newacronym{rag}{RAG}{Retrieval-Augmented Generation}
\newacronym{rbac}{RBAC}{Role-Based Access Control}
\newacronym{rdf}{RDF}{Resource Description Framework}
\newacronym{rdfs}{RDFS}{RDF Schema}
\newacronym{rest}{REST}{Representational State Transfer}
\newacronym{rlhf}{RLHF}{Reinforcement Learning from Human Feedback}
\newacronym{sdk}{SDK}{Software Development Kit}
\newacronym{shacl}{SHACL}{Shapes Constraint Language}
\newacronym{sparql}{SPARQL}{SPARQL Protocol and RDF Query Language}
\newacronym{spin}{SPIN}{SPARQL Inferencing Notation}
\newacronym{sft}{SFT}{Supervised Fine-Tuning}
\newacronym{tls}{TLS}{Transport Layer Security}
\newacronym{ttl}{TTL}{Time to Live}
\newacronym{ui}{UI}{User Interface}
\newacronym{url}{URL}{Uniform Resource Locator}
\newacronym{uri}{URI}{Uniform Resource Identifier}
\newacronym{ux}{UX}{User Experience}
\newacronym{vbac}{VBAC}{View-Based Access Control}
\newacronym{wthreec}{W3C}{World Wide Web Consortium}
\newacronym{xacml}{XACML}{eXtensible Access Control Markup Language}
\newacronym{webid}{WebID}{Web Identity}
\newacronym{wac}{WAC}{Web Access Control}
\newacronym{odrl}{ODRL}{Open Digital Rights Language}
\newacronym{yasgui}{YASGUI}{Yet Another SPARQL Graphical User Interface}
% \input{Header/Glossar}


\areaset[0.5cm]{16cm}{24cm}