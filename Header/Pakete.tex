% ============ Sprache & Kodierung ============
\usepackage[ngerman]{babel}
\usepackage[utf8]{inputenc}     % statt latin1
\usepackage[T1]{fontenc}
\usepackage{textcomp}           % Euro & Co. (bei neuen LaTeX-Installationen oft entbehrlich)

% \RequirePackage[ngerman=ngerman-x-latest]{hyphsubst} % meist nicht mehr nötig

% ============ Mathe ============
\usepackage{amsmath,amsfonts}

% ============ Layout ============
\usepackage{setspace}
\usepackage{geometry}

% ============ Grafiken & Float-Handling ============
\usepackage{graphicx}
\usepackage{wrapfig}            % statt veraltetem floatflt
% \usepackage{pstricks}         % -> vermeiden bei pdfLaTeX; alternativ TikZ:
% \usepackage{tikz}

% ============ Farben ============
\usepackage[dvipsnames]{xcolor} % statt color
% Beispiel-Farben, falls du sie in hyperref nutzt:
\definecolor{InterneLinkfarbe}{RGB}{0,0,120}
\definecolor{ExterneLinkfarbe}{RGB}{0,100,0}

% ============ Tabellen ============
\usepackage{array}
\usepackage{longtable}
\usepackage{booktabs}         

% ============ Quellcode ============
\usepackage{verbatim}
\usepackage{moreverb}
\usepackage{listings}
\lstloadlanguages{Java,HTML,bash}
\lstdefinelanguage{SPARQL}{
  morekeywords={SELECT,ASK,CONSTRUCT,DESCRIBE,WHERE,FROM,GRAPH,OPTIONAL,FILTER,
    LIMIT,OFFSET,ORDER,BY,GROUP,HAVING,VALUES,UNION,BIND,SERVICE,MINUS,INSERT,
    DELETE,DATA,WITH,USING,LOAD,CLEAR,CREATE,DROP,ADD,MOVE,COPY},
  sensitive=true,
  morecomment=[l]{#},
  morestring=[b]"
}
\lstset{
  language=SPARQL,
  frame=tb,
  basicstyle=\footnotesize\ttfamily,
  showstringspaces=false,
  keywordstyle=\bfseries,
  commentstyle=\itshape,
  xleftmargin=5pt, xrightmargin=5pt,
  aboveskip=\bigskipamount, belowskip=\bigskipamount
}

% ============ Zitate ============
\usepackage[autostyle=true]{csquotes}

% ============ Sonstiges ============
\usepackage{xspace}
\usepackage{caption}
\usepackage{scrhack}
\usepackage{microtype}

% ============ Links (nahe ans Ende) ============
\usepackage[
  pdftex,
  colorlinks=true,
  linkcolor=InterneLinkfarbe,
  urlcolor=ExterneLinkfarbe
]{hyperref}
% \usepackage[all]{hypcap}  % i. d. R. nicht mehr nötig
% \usepackage{url}          % wird von hyperref abgedeckt

% ============ Glossar (nach hyperref laden) ============
\usepackage[
  nonumberlist,
  acronym,
  toc
]{glossaries}

